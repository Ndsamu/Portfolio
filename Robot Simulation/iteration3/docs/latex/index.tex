\hypertarget{index_intro_sec}{}\section{User Guide}\label{index_intro_sec}
  \hypertarget{index_introduction}{}\subsection{Introduction}\label{index_introduction}
This project represents a simple user-\/configured \hyperlink{classRobot}{Robot} simulation in which \hyperlink{classRobot}{Robot} entities react to two other types of entities using specialized sensors, \hyperlink{classLight}{Light} and \hyperlink{classFood}{Food}. The user can specify the quantity of each entity as well as various other parameters to effect the simulation. Robots exhibit three different types of behavior\+: Fear, Exploratory, and Aggresive (when hungry). The Fear and Exploratory behaviors are reactions to lights and can be configured by the user, however the Aggressive behavior is only exhibited when the \hyperlink{classRobot}{Robot} experiences hunger. \hyperlink{classRobot}{Robot} behavior is calculated by the connection of a \hyperlink{classRobot}{Robot}\textquotesingle{}s left and right Sensors (Light/\+Food) and the velocities of its left and right wheels. After 30 second without a \hyperlink{classRobot}{Robot} coming into contact with a \hyperlink{classFood}{Food} entity, it will begin to act increasingly aggressive. After a total of two minutes, the \hyperlink{classRobot}{Robot} will ignore lights altogether in favor of \hyperlink{classFood}{Food}. If a \hyperlink{classRobot}{Robot} has not come into contact with \hyperlink{classFood}{Food} after a total of two minutes and 30 seconds, the \hyperlink{classRobot}{Robot} will starve and the simulation will end with the game being paused. At this point, the user must click New Game to reset the \hyperlink{classArena}{Arena}. Upon a new game, the user is allowed to once again Configure the \hyperlink{classArena}{Arena} which will take effect after clicking the Play button. A user can also begin a new game whenever they please by clicking the New Game button. If any \hyperlink{classRobot}{Robot} comes into contact with another \hyperlink{classRobot}{Robot}, a wall, or a \hyperlink{classFood}{Food} entity it will reverse in an arc momentarily before continuing as usual. The same behavior is exhibited by lights except that a \hyperlink{classLight}{Light} will only collide with walls because they hover above the \hyperlink{classArena}{Arena} and other entities. The simulation will continue indefinitely so long as none of the Robots have starved.\hypertarget{index_non_technical}{}\subsection{General User}\label{index_non_technical}
{\bfseries Play/\+Pause Button}

As mentioned previously, a user can pause the simulation at their convenience by clicking the Pause button. When the simulation begins, it will default to being paused and the user is required to click the Play button upon which the \hyperlink{classArena}{Arena} Configurations will take effect.

{\bfseries New Game Button}

The New Game button will reset the \hyperlink{classArena}{Arena} by removing all entities and pausing the simulation. At this point a user can adjust the \hyperlink{classArena}{Arena} configurations and click the Play button for the configurations to take effect as the simulation begins.

{\bfseries \hyperlink{classRobot}{Robot} Quantity Slider}

This slider is somewhat self-\/explanatory and allows the user to configure the number of Robots in a range of 0-\/10. With a selection of 0, the simulation will continue indefinitely.

{\bfseries \hyperlink{classRobot}{Robot} Ratio Slider (Fear vs Exploratory)}

The \hyperlink{classRobot}{Robot} ratio controls the number of Robots that exhibit Fear behavior versus the number of Robots that exhibit Exploratory behavior (explanations follow). The ratio is, of course, relative to the number of Robots configured by the user.

{\bfseries Fear Behavior}

A \hyperlink{classRobot}{Robot} exhibiting Fear behavior will actively avoid lights by steering the other direction. This is accomplished by connecting the left \hyperlink{classLightSensor}{Light\+Sensor} to the left wheel\textquotesingle{}s velocity and the right \hyperlink{classLightSensor}{Light\+Sensor} to the right wheel\textquotesingle{}s velocity. For example, a \hyperlink{classRobot}{Robot} will accelerate away from a \hyperlink{classLight}{Light} on its left because the left \hyperlink{classSensor}{Sensor} reading is greater than the right \hyperlink{classSensor}{Sensor} reading so the \hyperlink{classRobot}{Robot} naturally turns to the right away from the \hyperlink{classLight}{Light}. The only situation in which a \hyperlink{classRobot}{Robot} will not avoid a \hyperlink{classLight}{Light} is if it senses the \hyperlink{classLight}{Light} as it is directly in front of the \hyperlink{classRobot}{Robot} because the left and right \hyperlink{classSensor}{Sensor} readings will be equal, thus accelerating straight towards the \hyperlink{classLight}{Light}.

{\bfseries Exploratory Behavior}

A \hyperlink{classRobot}{Robot} exhibiting Exploratory behavior will \char`\"{}explore\char`\"{} whichever \hyperlink{classLight}{Light} is closest to the \hyperlink{classRobot}{Robot}. This is accomplished by connecting the left \hyperlink{classLightSensor}{Light\+Sensor} reading to the right wheel\textquotesingle{}s velocity with an inverted connection and the right \hyperlink{classLightSensor}{Light\+Sensor} reading to the left wheel\textquotesingle{}s velocity with an inverted connection. For example, if a \hyperlink{classLight}{Light} is on the left side of a \hyperlink{classRobot}{Robot}, the the left \hyperlink{classSensor}{Sensor}\textquotesingle{}s reading will be greater than the right \hyperlink{classSensor}{Sensor}\textquotesingle{}s reading as stated before. However, because the left \hyperlink{classSensor}{Sensor} is connected to the right wheel\textquotesingle{}s velocity, it will increase it\textquotesingle{}s right wheel\textquotesingle{}s velocity until the left and right Sensors reach and equilibrium and the \hyperlink{classRobot}{Robot} accelerates directly backwards toward the \hyperlink{classLight}{Light}. The backwards acceleration results from the inverted connections which gives a negative velocity to the left and right wheel velocities.

{\bfseries Aggressive Behavior}

This behavior is simply the Exploratory behavior backwards. Rather than an inverted connection between sensors and wheel velocities, the \hyperlink{classRobot}{Robot} has a direct connection between the left \hyperlink{classLightSensor}{Light\+Sensor} and the right wheel\textquotesingle{}s velocity as well as the right \hyperlink{classLightSensor}{Light\+Sensor} and the left wheel\textquotesingle{}s velocity. Thus, the \hyperlink{classRobot}{Robot} demonstrates the Exploratory behavior while accelerating directly towards a \hyperlink{classLight}{Light} rather than backwards towards it.

{\bfseries \hyperlink{classLight}{Light} Quantity Slider}

The \hyperlink{classLight}{Light} Quantity slider allows the user to configure the number of lights in the \hyperlink{classArena}{Arena} within a range of 0-\/5.

{\bfseries \hyperlink{classLight}{Light} Sensitivity Slider}

The \hyperlink{classLight}{Light} Sensitivity slider allows the user to configure the Sensitivity of all \hyperlink{classRobot}{Robot}\textquotesingle{}s \hyperlink{classLight}{Light} sensors in a range from 0-\/100\%. At 0\%, the Robots will ignore lights completely. At 100\% \hyperlink{classLight}{Light} sensitivity the Robots will react to lights if they are within a distance of roughly one third of the length of the \hyperlink{classArena}{Arena}.

{\bfseries \hyperlink{classFood}{Food} Quantity Slider}

The \hyperlink{classFood}{Food} Quantity slider allows the user to configure the number of food entities in the \hyperlink{classArena}{Arena} within a range of 0-\/5.

{\bfseries \hyperlink{classFood}{Food} Button (O\+N/\+O\+FF)}

The \hyperlink{classFood}{Food} Button allows the user to toggle on or off the presence of \hyperlink{classFood}{Food}. This is the only \hyperlink{classArena}{Arena} configuration that takes effect without the need to start a new game. While it will default to ON, the user can turn off \hyperlink{classFood}{Food} at any point by clicking the \hyperlink{classFood}{Food} (ON) button which will then switch to \hyperlink{classFood}{Food} (O\+FF). Upon \hyperlink{classFood}{Food} being turned off, all \hyperlink{classFood}{Food} entities will disappear and Robots will not experience hunger. Thus, the simulation will continue indefinitely without starvation of Robots. When \hyperlink{classFood}{Food} is turned back on, the \hyperlink{classFood}{Food} entities will appear randomly placed in the \hyperlink{classArena}{Arena} and Robots will begin to experience hunger once again (with a reset hunger countdown).\hypertarget{index_technical}{}\subsection{Technical User}\label{index_technical}
{\bfseries Flow of Control}

For the technical user, a more detailed explantion follows.

The structure of the simulation begins with the \hyperlink{main_8cc}{main.\+cc} file which instantiates a \hyperlink{classController}{Controller} and executes its Run() method. This class instantiates an \hyperlink{classArena}{Arena} and \hyperlink{classGraphicsArenaViewer}{Graphics\+Arena\+Viewer} upon construction and calls the \hyperlink{classGraphicsArenaViewer}{Graphics\+Arena\+Viewer}\textquotesingle{}s Run method which begins the simulation. The \hyperlink{classArena}{Arena} is left empty until the user clicks Play where the \hyperlink{classGraphicsArenaViewer}{Graphics\+Arena\+Viewer} calls the appropriate methods through the controller to initialize the \hyperlink{classArena}{Arena} with the given parameters. The controller calls the \hyperlink{classArena}{Arena}\textquotesingle{}s methods for each entity with the appropriate quantity (set by the user) which in turn calls the \hyperlink{classEntityFactory}{Entity\+Factory}\textquotesingle{}s methods for instantiation of each entity. Entities are then stored in various respective vectors within the \hyperlink{classArena}{Arena} class. The \hyperlink{classRobot}{Robot} Ratio is used in the controller to instantiate the proper quantity of each \hyperlink{classRobot}{Robot} Type in the \hyperlink{classArena}{Arena}. The \hyperlink{classLight}{Light} Sensitivity is passed with the \hyperlink{classRobot}{Robot} Type to each parameter which sets each \hyperlink{classRobot}{Robot}\textquotesingle{}s member variable for \hyperlink{classLight}{Light} Sensitivity. Each robot has a left and right \hyperlink{classSensor}{Sensor} for both \hyperlink{classFood}{Food} and Lights. These sensors connect with the wheel velocities via a method within the \hyperlink{classRobot}{Robot}\textquotesingle{}s \hyperlink{classMotionHandlerRobot}{Motion\+Handler\+Robot} called Update\+Velocity. Update\+Velocity is effectively where the behavior of a \hyperlink{classRobot}{Robot} is determined. A switch statement within this method calculates the appropriate velocity given the \hyperlink{classRobot}{Robot} type.

Due to poor time management, this is where the technical user guide will reach its conclusion. Thank you for your time. Have a great summer! 